\documentclass[letterpaper]{article} 

\setlength{\pdfpagewidth}{\paperwidth}
\setlength{\pdfpageheight}{\paperheight}

\usepackage{fullpage}


% scoopversion.tex is generated automatically to define \scoopversion
\include{scoop.version}


\begin{document}
\title{SCOOP: Source COmpilation Optimizations for Parallelism \scoopversion}
\date{last update: 29/05/11}
\maketitle

\section{License}

Copyright (c) 2010, \\
 Dimitris Hassapis  <hassapis@ics.forth.gr>\\
 Foivos Zakkak        <zakkak@ics.forth.gr>\\
 Polyvios Pratikakis  <polyvios@ics.forth.gr>\\
All rights reserved.\\

Redistribution and use in source and binary forms, with or without\\
modification, are permitted provided that the following conditions are\\
met:\\

1. Redistributions of source code must retain the above copyright\\
notice, this list of conditions and the following disclaimer.\\

2. Redistributions in binary form must reproduce the above copyright\\
notice, this list of conditions and the following disclaimer in the\\
documentation and/or other materials provided with the distribution.\\

3. The names of the contributors may not be used to endorse or promote\\
products derived from this software without specific prior written\\
permission.\\

THIS SOFTWARE IS PROVIDED BY THE COPYRIGHT HOLDERS AND CONTRIBUTORS "AS\\
IS" AND ANY EXPRESS OR IMPLIED WARRANTIES, INCLUDING, BUT NOT LIMITED\\
TO, THE IMPLIED WARRANTIES OF MERCHANTABILITY AND FITNESS FOR A\\
PARTICULAR PURPOSE ARE DISCLAIMED. IN NO EVENT SHALL THE COPYRIGHT OWNER\\
OR CONTRIBUTORS BE LIABLE FOR ANY DIRECT, INDIRECT, INCIDENTAL, SPECIAL,\\
EXEMPLARY, OR CONSEQUENTIAL DAMAGES (INCLUDING, BUT NOT LIMITED TO,\\
PROCUREMENT OF SUBSTITUTE GOODS OR SERVICES; LOSS OF USE, DATA, OR\\
PROFITS; OR BUSINESS INTERRUPTION) HOWEVER CAUSED AND ON ANY THEORY OF\\
LIABILITY, WHETHER IN CONTRACT, STRICT LIABILITY, OR TORT (INCLUDING\\
NEGLIGENCE OR OTHERWISE) ARISING IN ANY WAY OUT OF THE USE OF THIS\\
SOFTWARE, EVEN IF ADVISED OF THE POSSIBILITY OF SUCH DAMAGE.\\

\section{Annotation Syntax}

\begin{verbatim}
    #pragma css wait all
    #pragma css wait on(list of expressions) //Not implemented yet
    #pragma css task [input(<input parameters>)] [inout(<input parameters>)] [output(<input parameters>)] [highpriority]
    (highpriority is simply ignored)

    Parameter notation:
      Non stride:
        <parameter>[\[parameter size\]]
      
      Stride:
        <parameter>\[stride]\[ els | elsz \]
        where stride is ... (TODO)
        els is ... (TODO)
        elsz is ... (TODO)

      The parameter size/stride/els/elsz must be an expression, thus
      we don't allow function calls. Also there is no support for the
      conditional operator (? :)
      

    Example:
      #pragma css task input(a, b[4*sizeof(b_type)]) output(c[16])
\end{verbatim}


\section{Dependencies}

\begin{enumerate}
  \item ocaml
  \item camlp4/ocaml-camlp4/ocaml-camlp4-devel
  \item flex
  \item bison
  \item indent
  \item ncurses-devel
  \item emacs
\end{enumerate}

for CELL\\
  you will also need the cell development package including ppu\_intrinsics.h\\
  ,altivec.h etc in your include path (C\_INCLUDE\_PATH)\\
  (e.g. export C\_INCLUDE\_PATH=$C\_INCLUDE\_PATH:/opt/cell/toolchain/lib/gcc/ppu/4.1.1/include:/opt/cell/toolchain/lib/gcc/spu/4.1.1/include )\\
  and ppu32-gcc in your PATH\\
  (e.g. export PATH=$PATH:/opt/cell/toolchain/bin )\\

\section{Install}

  ./configure \&\& make

\section{Run (Needs update)}

  scoop --arch=<x86/cell> [options] <file> [file2 ...]

\subsection{Options (Needs update)}

  --arch  Defines the target architecture (x86/cell)\\

  --debug Enable debugging information printing.\\

  --out-name Specify the output files' prefix.\\
          e.g. (default: final) will produce \\
          final.c and final\_func.c for the cell architecture\\
\textbf{FIXME}         and final.c? for the x86 architecture\\

\textbf{FIXME} to god x86 exei queue?\\
  --queue-size\\
          Specify the queue size for Cell. The value must match the one defined\\
          in the Makefile as MAX\_QUEUE\_ENTRIES\\

  --block-size\\
          Specify the block size for x86.  The value must match the one defined\\
          in the Makefile as BLOCK\_SZ\\

  --with-stats\\
          Enable code for statistics (for use with -DSTATISTICS)\\

  --with-unaligned-arguments\\
          Allow unalligned arguments in x86\\
          (for use with -DUNALIGNED\_ARGUMENTS\_ALLOWED)\\

  --threaded\\
          Generate thread safe code (for use with -DTPC\_MULTITHREADED)\\

\end{document}
