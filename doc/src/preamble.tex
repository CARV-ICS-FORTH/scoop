%%%%%%%%%%%%%%%%%%%%%%%%%%%%%%%%%%%%%%%%%%%%%%%%%%%%%%%%%%%%%%%%%%%%%%%%%%%%%%%%
% Include packages
%%%%%%%%%%%%%%%%%%%%%%%%%%%%%%%%%%%%%%%%%%%%%%%%%%%%%%%%%%%%%%%%%%%%%%%%%%%%%%%%
% most of the packages bellow are from
% http://tex.stackexchange.com/questions/553/what-packages-do-people-load-by-default-in-latex

% less hyphens... (works only with pdflatex)
% stretch=10 allows font expansion up to 1% (default is 2%)
\usepackage[stretch=10]{microtype}
% disable ff and fi etc. ligatures
\DisableLigatures{encoding = *, family = *}
% allow breakable urls
\usepackage[hyphens]{url}
% If grouped citations are in the same order put the range i.e. 5-7
% DOES NOT WORK WITH biblatex package
%\usepackage{cite}
% For hyperlinks and more in the pdfs
\usepackage{hyperref}
\hypersetup{
  linktoc=all,
  bookmarks=false,           % show bookmarks bar?
  bookmarksopen,
  bookmarksnumbered,
  colorlinks = true,
%  linkcolor=black,           % color of interlinks
  citecolor=black,           % color of the citation links
%  urlcolor=black,            % the url color
  unicode=true,              % non-Latin characters in Acrobat’s bookmarks
  pdftoolbar=true,           % show Acrobat’s toolbar?
  pdfmenubar=true,           % show Acrobat’s menu?
  pdffitwindow=true,         % window fit to page when opened
  pdftitle={greenvm User's Manual},  % title
  pdfborder={ 0 0 0 },        % uBorder tin links
  pdfauthor = {Foivos S. Zakkak},
  pdfcreator = {Foivos S. Zakkak},
}
% handles spaces in commands (i.e. \newcommand{\test}[0]{Test\xspace},
% this way \test will be replaced by Test and a space *iff* needed)
\usepackage{xspace}
% For citations and bibliographies, biblatex is the package of my
% choice
\usepackage{biblatex}
% The booktabs package creates much nicer looking tables than the
% standard latex tables;
\usepackage{booktabs}
% the array package's ability to create custom columns is invaluable
% for formatting tabular material on a per-column basis.
\usepackage{array}
% For unicode files
\usepackage[utf8]{inputenc}
% For including figures, rotating or scaling text. I also use the
% \graphicspath command to specify a subfolder to help organize my
% figures and so I can easily change between, for example, a set of
% figures for internal used (with extra info) and final versions for
% distribution.
\usepackage{graphicx}
% define the default path for the figures
\graphicspath{{./figs/}}
% This package provides very sophisticated facilities for reading and
% writing verbatim TeX code. Users can perform common tasks like
% changing font family and size, numbering lines, framing code
% examples, colouring text and conditionally processing text.
\usepackage{fancyvrb}

% Using listings
\usepackage{listings}
% Change the listing captions
% to use \begin{lstlisting}[label=some-code,caption=Some Code]
\usepackage{caption}
\DeclareCaptionFont{white}{\color{white}}
\DeclareCaptionFormat{listing}{\colorbox{MidnightBlue}{\parbox{.98\textwidth}{#1#2#3}}}
\captionsetup[lstlisting]{format=listing,labelfont=white,textfont=white}
\renewcommand{\lstlistingname}{Code}
\renewcommand\lstlistlistingname{List of Codes}
\def\lstlistingautorefname{Code }
% lstset
\lstset{
  language=C,
  basicstyle=\sffamily,
  columns=flexible,
  numbers=left,
%   numberstyle=\em\scriptsize,s
  showstringspaces=false,
  alsoletter={-},
  morekeywords={in,out,inout,pragma,task,wait,all,safe,scoop,input,output,start,finish},
  otherkeywords={INOUT,INPUT,SAFE,OUTPUT,START},
  literate={-}{-}1,
  numbersep=1em,
  xleftmargin=2.5em,
  xrightmargin=1em,
  escapeinside={@}{@},
  morecomment=[l][\bf\color{BrickRed}]{\#pragma\ scoop},
  commentstyle=\color{gray},
%   identifierstyle=\color{green},
%   backgroundcolor=\color{gray},
  keywordstyle=\bf\color{MidnightBlue},
  stringstyle=\color{OliveGreen},
}
% Allow different numbering styles in enumerate environment
\usepackage{enumerate}
% Adds the \todo command which adds nice todo blocks to the document
\usepackage{todonotes}
% change the side for the todos
\reversemarginpar
% I much prfeer no indentation and space between paragraphs
\usepackage[parfill]{parskip}
% Add fonts
\usepackage{comfortaa}
% use tikz
\usepackage{tikz}
% For quotations
\usepackage[
indentfirst=true,
font=itshape,
begintext=\textquotedblleft,
endtext=\textquotedblright,
]{quoting}
%%%%%%%%%%%%%%%%%%%%%%%%%%%%%%%%%%%%%%%%%%%%%%%%%%%%%%%%%%%%%%%%%%%%%%%%%%%%%%%%

% Change the margins
\usepackage[margin=2.5cm]{geometry}
\setlength{\marginparwidth}{2cm}


%% Boxes for notes and caution messages
\usetikzlibrary{shapes}
\tikzstyle{notebox} = [draw=yellow, fill=yellow!10, very thick,
    rectangle, rounded corners, inner sep=10pt, inner ysep=15pt]
\tikzstyle{notetitle} =[draw=yellow, fill=white, text=black, very thick]

\newcommand{\notebox}[1]{
  \begin{tikzpicture}
    \node [notebox] (box){%
      \begin{minipage}{.95\linewidth}
        #1
      \end{minipage}
    };
    \node[notetitle, rounded corners, right=10pt] at (box.north west) {Note:};
  \end{tikzpicture}%
}

\tikzstyle{cautionbox} = [draw=red, fill=red!10, very thick,
    rectangle, rounded corners, inner sep=10pt, inner ysep=15pt]
\tikzstyle{cautiontitle} =[draw=red, fill=white, text=black, very thick]

\newcommand{\cautionbox}[1]{
  \begin{tikzpicture}
    \node [cautionbox] (box){%
      \begin{minipage}{.95\linewidth}
        #1
      \end{minipage}
    };
    \node[cautiontitle, rounded corners, right=10pt] at (box.north west) {\textbf{Caution:}};
  \end{tikzpicture}%
}

%%

\renewcommand{\FancyVerbFormatLine}[1]{%
  \$ #1}
\DefineVerbatimEnvironment{bash}{Verbatim}{
  fontsize=\footnotesize,
  frame=lines,
  framesep=3mm,
  label={\normalsize{Code}},
  labelposition=topline,
  commentchar=\#,
}
